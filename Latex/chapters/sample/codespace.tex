\section{Codeblocks}
There are several ways to embed code in a \LaTeX{} file.
Here I demonstrate inline code, embedded codeblocks, and external import.

This version of embedded code uses a custom environment defined in the main file.
You can expand this environment instead of using \mintinline{latex}{code} as well.

\begin{code}{Python}
  class iostream:
      def __lshift__(self, other):
          print(other, end='')
          return self

      def __repr__(self):
          return ''


  if __name__ == "__main__":
      cout = iostream()
      endl = '\n'
      cout << "Hello" << ", " << "World!" << endl
\end{code}

\begin{mdframed}[leftline=false,rightline=false,backgroundcolor=magenta!10,nobreak=false]
  \inputminted[linenos=true,breaklines,breaksymbolleft=,obeytabs=true,tabsize=2]{Python}{assets/main.py}
\end{mdframed}

You can also define your custom inline as \url{https://tex.stackexchange.com/a/148479}.

This is one way to input algorithms.

\begin{algorithm}
  \caption{QL algorithm}
  Initialize \(Q\)-table values \((Q(s, a))\) arbitrarily\;
  Initialize a state \((s_t)\)\;
  Repeat Steps~\ref{alg:step_4} to~\ref{alg:step_6} until learning period ends\;
  Choose an action \((a_t)\) for the current state \((s_t)\) using an exploratory policy\; \nllabel{alg:step_4}
  Take action \((a_t)\) and observe the new state \((s_t + 1)\) and reward \((r_t + 1)\)\;
  Update \(Q\)-value\; \nllabel{alg:step_6}
\end{algorithm}
